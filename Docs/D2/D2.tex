% !TeX root = ./D2.tex
\documentclass[]{article}

% Imported Packages
%------------------------------------------------------------------------------
\usepackage{amssymb}
\usepackage{amstext}
\usepackage{amsthm}
\usepackage{amsmath}
\usepackage{enumerate}
\usepackage{fancyhdr}
\usepackage[margin=1in]{geometry}
\usepackage{graphicx}
\usepackage{extarrows}
\usepackage{setspace}
\usepackage{float}
%------------------------------------------------------------------------------

% Header and Footer
%------------------------------------------------------------------------------
\pagestyle{plain}  
\renewcommand\headrulewidth{0.4pt}                                      
\renewcommand\footrulewidth{0.4pt}                                    
%------------------------------------------------------------------------------

% Title Details
%------------------------------------------------------------------------------
\title{
    \textbf{Group 10 - Deliverable \#2}\\
    \large SFWRENG 3A04: Software Design III - Large System Design
}
\author{
    Andrew Hum 400138826\\
    Arkin Modi 400142497\\
    Hongzhao Tan 400136957\\
    Christopher Vishnu 400129743\\
    Shengchen Zhou 400050783\\
}
\date{March 5, 2020}
%------------------------------------------------------------------------------

% Document
%------------------------------------------------------------------------------
\begin{document}

\maketitle
\newpage

\section{Introduction}
\label{sec:introduction}

\subsection{Purpose}
\label{sub:purpose}
The purpose of the document is to focus on the architecture of the HackerSim 
system. The system's architecture is based upon business events developed in 
Deliverable 1 to outline the components of the HackerSim software for both the 
client and the developer. It covers the architectural decisions that have been 
made regarding the system and its components. This document is intended for the 
project manager, the current project team and any future development teams for 
the HackerSim Project.

\subsection{System Description}
\label{sub:system_description}
The HackerSim system is an interactive game that will allow the user to raise a 
Software Engineer in their room. The main component of our software would be 
the General Room which is the link that interacts with the rest of the 
sub-components. The main sub-components that the General Room interacts with 
which would be the Shop, Friends and Chat, Project and the Time-step. The Shop 
component focuses on interacting with the inventory for purchasing and browsing 
items. The Friends and Chat component focuses on providing message functionality 
between the user and the friends. The Project component focuses on the project 
and future projects the Software Engineer has to do to gain in-game currency. 
Finally, the Time-step component focuses on the passage of time and which 
affects the Software Engineer’s attributes.

\subsection{Overview}
\label{sub:overview}
This document is organized by the following sections: Analysis Class Diagram, 
Architectural Design, Class Responsibility Collaboration Cards. Analysis Class 
Diagram focuses on providing details about the structure of the classes and 
their relationships. Architectural Design focuses on the overall architectural 
design of the HackerSim application, showing the division of the system into 
subsystems. Finally, Class Responsibility Collaboration (CRC) Cards focus on 
each individual class and its responsibilities and relations in which they 
collaborate with other classes.

\subsection{Definitions, Acronyms, Abbreviations}
\label{sub:definitions_acronyms_abbreviations}
SE - Software Engineer

\section{Analysis Class Diagram}
\label{sec:analysis_class_diagram}
\begin{figure}[H]
    \centering
    \includegraphics[scale=0.4]{"Analysis Class Diagram".png}
    \caption{Analysis Class Diagram for Hacker Sim}
\end{figure}


\section{Architectural Design}
\label{sec:architectural_design}

\subsection{System Architecture}
\label{sub:system_architecture}
The architecture style that we will be using for our system is an architecture 
that is interaction-oriented; Presentation-Abstraction-Control (PAC). Among the 
other architectural styles, data-centric and data-flow architecture, we felt 
that an interaction-oriented architecture would fit our project the best. We 
believe this because this type of architecture is most appropriate when there 
is a need to separate the user’s interactions from the data abstraction and 
data processing. This type of architecture effectively divides the data into 
logical sections and allows an effective presentation of content based on 
changes in the data utilizing the view modules. From interaction-oriented 
architecture, there were two possible routes to take, Model-View-Controller 
(MVC) architecture and Presentation-Abstraction-Control (PAC) architecture. We 
chose to use Presentation-Abstraction-Control as it emphasizes a hierarchical 
architecture that allows low-coupling of agents protecting other agents from 
being modified by the changes of one. In essence, this will provide us with an 
effective way to identify our core entities and link them to relevant agents 
with ease. It will also allow multiple agents to perform their tasks, and update 
their view accordingly without affecting the other agents. MVC highly depends 
upon a uniform data set for all views that changes at a constant rate. Given that 
we have several different entities, each with their unique data set that will 
be updated at different rates, the need for PAC architecture for our system is 
evident.

\begin{figure}[H]
    \centering
    \includegraphics[scale=0.5]{"PAC Structural Architecture Diagram".png}
    \caption{PAC Structural Architecture Diagram}
\end{figure}

\subsection{Subsystems}
\label{sub:subsystems}
The following are explanations of the subsystems for our project that outline 
their purpose and relationship to other subsystems.
\begin{enumerate}
    \item General (Room)\\
    Purpose: This agent is the highest-level agent that controls every other 
    agent and presents the default ‘room’ view as well as ability to interact 
    with the objects, shop, friends and the SE\\
    Direct Relationships: Shop, Friends and Chat, Projects    
    \item Shop\\
    Purpose: This agent is an intermediate-level agent that controls the user’s 
    interactions with the shop, including purchasing and browsing items. This 
    agent may then update the SE’s inventory if the user chooses to purchase an 
    item.\\
    Direct Relationships: SE Inventory, User Attributes, General (Room)
    \item Friends \& Chat\\
    Purpose: This agent is an intermediate-level agent that stores data on chat 
    history, and provides the user with the ability to message friends based on 
    the user’s friends list.\\
    Direct Relationships: User’s Attributes, General (Room)
    \item Projects\\
    Purpose: This intermediate-level agent stores information regarding current 
    and future projects the SE may work on to gain currency.\\
    Direct Relationships: User’s Attributes, SE Attributes, General (Room), 
    Time-Step
    \item Time-Step\\
    Purpose: This intermediate-level agent controls the passage of time, and 
    signals the SE Attributes accordingly with changes in their current state 
    or information. These changes include hunger, project completion, tiredness 
    level, happiness change, etc.\\
    Direct Relationships: SE Attributes, Projects
    \item Software Engineer (SE) Inventory\\
    Purpose: This bottom-level agent stores data based on the items in the SE’s 
    inventory and adjusts the data and view based upon the user’s interactions 
    with the store or through the use of items.\\
    Direct Relationships: Shop, SE Attributes
    \item Software Engineer (SE) Attributes\\
    Purpose: This bottom-level agent stores data based on the SE’s current 
    attributes such as their happiness index, how tired they are, if they are 
    hungry, if they completed a project, etc. These attributes may be adjusted 
    by the Time-Step agent or the Project agent, or changed directly by the 
    user.\\
    Direct Relationships: Projects, Time-Step, SE Inventory
    \item User Attributes\\
    Purpose: This bottom-level agent stores data based on the user’s current 
    attributes. Some data is immutable and may not be changed throughout the 
    game, however, there is data that may be manipulated by the project 
    controller, providing the user with skill points to increase the SE’s 
    ability, a change in their current balance, and addition or removal or 
    friends.\\
    Direct Relationships: Shop, Friends and Chat, Projects
\end{enumerate}
	
\section{Class Responsibility Collaboration (CRC) Cards}
\label{sec:class_responsibility_collaboration_crc_cards}
\subsection{Entity Class CRC Cards}
\label{sec:entity_class_CRC_cards}

\begin{table}[H]
    \centering
    \begin{tabular}{|p{5cm}|p{5cm}|}
        \hline
        \multicolumn{2}{|l|}{\textbf{Class Name:} General Room}\\
        \hline
        \textbf{Responsibility:} & \textbf{Collaborators:}\\
        \hline
        Store SE current state (gaming, working, exercising, sleeping, etc) & ~\\
        ~ & ~\\
        Store room items (exercise equipment, furniture) & ~\\
        ~ & ~\\
        Store room size & ~\\
        \hline
    \end{tabular}
\end{table}

\begin{table}[H]
    \centering
    \begin{tabular}{|p{5cm}|p{5cm}|}
        \hline
        \multicolumn{2}{|l|}{\textbf{Class Name:} User Attributes}\\
        \hline
        \textbf{Responsibility:} & \textbf{Collaborators:}\\
        \hline
        Store currency & ~\\
        ~ & ~\\
        Store coding experience level & ~\\
        ~ & ~\\
        Store friends list & ~\\
        \hline
    \end{tabular}
\end{table}

\begin{table}[H]
    \centering
    \begin{tabular}{|p{5cm}|p{5cm}|}
        \hline
        \multicolumn{2}{|l|}{\textbf{Class Name:} SE Attributes}\\
        \hline
        \textbf{Responsibility:} & \textbf{Collaborators:}\\
        \hline
        Store SE’s name and sex & ~\\
        ~ & ~\\
        Store SE’s current state (happiness, tiredness, hunger, etc.) & ~\\
        \hline
    \end{tabular}
\end{table}

\begin{table}[H]
    \centering
    \begin{tabular}{|p{5cm}|p{5cm}|}
        \hline
        \multicolumn{2}{|l|}{\textbf{Class Name:} SE Inventory}\\
        \hline
        \textbf{Responsibility:} & \textbf{Collaborators:}\\
        \hline
        Store SE’s items and quantities & ~\\
        \hline
    \end{tabular}
\end{table}

\begin{table}[H]
    \centering
    \begin{tabular}{|p{5cm}|p{5cm}|}
        \hline
        \multicolumn{2}{|l|}{\textbf{Class Name:} Shop}\\
        \hline
        \textbf{Responsibility:} & \textbf{Collaborators:}\\
        \hline
        Store items and prices & ~\\
        ~ & ~\\
        Store skills and prices & ~\\
        \hline
    \end{tabular}
\end{table}

\begin{table}[H]
    \centering
    \begin{tabular}{|p{5cm}|p{5cm}|}
        \hline
        \multicolumn{2}{|l|}{\textbf{Class Name:} Projects}\\
        \hline
        \textbf{Responsibility:} & \textbf{Collaborators:}\\
        \hline
        Store all projects & ~\\
        ~ & ~\\
        Store project information (name, deadline, reward, skills required, completion status) & ~\\
        \hline
    \end{tabular}
\end{table}

\begin{table}[H]
    \centering
    \begin{tabular}{|p{5cm}|p{5cm}|}
        \hline
        \multicolumn{2}{|l|}{\textbf{Class Name:} Time-Step}\\
        \hline
        \textbf{Responsibility:} & \textbf{Collaborators:}\\
        \hline
        Store time elapsed & ~\\
        \hline
    \end{tabular}
\end{table}

\begin{table}[H]
    \centering
    \begin{tabular}{|p{5cm}|p{5cm}|}
        \hline
        \multicolumn{2}{|l|}{\textbf{Class Name:} Chat History}\\
        \hline
        \textbf{Responsibility:} & \textbf{Collaborators:}\\
        \hline
        Store all chat history and username of the user communicating with  & ~\\
        \hline
    \end{tabular}
\end{table}

\subsection{Boundary Class CRC Cards}
\label{sec:boundary_class_CRC_cards}

\begin{table}[H]
    \centering
    \begin{tabular}{|p{5cm}|p{5cm}|}
        \hline
        \multicolumn{2}{|l|}{\textbf{Class Name:} General Room Interface}\\
        \hline
        \textbf{Responsibility:} & \textbf{Collaborators:}\\
        \hline
        Present SE current state (gaming, sleeping, exercising, working) & General Room Controller\\
        ~ & ~\\
        Present room based on size & General Room Controller\\
        ~ & ~\\
        Present room items (furniture, equipment, etc) & General Room Controller\\
        \hline
    \end{tabular}
\end{table}

\begin{table}[H]
    \centering
    \begin{tabular}{|p{5cm}|p{5cm}|}
        \hline
        \multicolumn{2}{|l|}{\textbf{Class Name:} User Information Interface}\\
        \hline
        \textbf{Responsibility:} & \textbf{Collaborators:}\\
        \hline
        Present username & User Attributes Controller\\
        ~ & ~\\
        Present user’s currency & User Attributes Controller\\
        ~ & ~\\
        Present coding experience level & User Attributes Controller\\
        ~ & ~\\
        Present user’s friends & User Attributes Controller\\
        \hline
    \end{tabular}
\end{table}

\begin{table}[H]
    \centering
    \begin{tabular}{|p{5cm}|p{5cm}|}
        \hline
        \multicolumn{2}{|l|}{\textbf{Class Name:} Settings Interface}\\
        \hline
        \textbf{Responsibility:} & \textbf{Collaborators:}\\
        \hline
        Present option to quit & General Room Controller\\
        \hline
    \end{tabular}
\end{table}

\begin{table}[H]
    \centering
    \begin{tabular}{|p{5cm}|p{5cm}|}
        \hline
        \multicolumn{2}{|l|}{\textbf{Class Name:} SE Creation UI}\\
        \hline
        \textbf{Responsibility:} & \textbf{Collaborators:}\\
        \hline
        Receive input for SE traits (name, sex) & SE Attributes Controller\\
        \hline
    \end{tabular}
\end{table}

\begin{table}[H]
    \centering
    \begin{tabular}{|p{5cm}|p{5cm}|}
        \hline
        \multicolumn{2}{|l|}{\textbf{Class Name:} SE Attribute UI}\\
        \hline
        \textbf{Responsibility:} & \textbf{Collaborators:}\\
        \hline
        Present SE Attributes (sex, name, happiness, hunger, skill, fitness level, tiredness level) & SE Attributes Controller\\
        \hline
    \end{tabular}
\end{table}

\begin{table}[H]
    \centering
    \begin{tabular}{|p{5cm}|p{5cm}|}
        \hline
        \multicolumn{2}{|l|}{\textbf{Class Name:} SE Inventory UI}\\
        \hline
        \textbf{Responsibility:} & \textbf{Collaborators:}\\
        \hline
        Present items list & SE Inventory Controller\\
        ~ & ~\\
        Receive input to use items & SE Inventory Controller\\
        ~ & ~\\
        Receive input to discard items & SE Inventory Controller\\
        ~ & ~\\
        Use Items & SE Inventory Controller\\
        \hline
    \end{tabular}
\end{table}

\begin{table}[H]
    \centering
    \begin{tabular}{|p{5cm}|p{5cm}|}
        \hline
        \multicolumn{2}{|l|}{\textbf{Class Name:} Exercise Equipment}\\
        \hline
        \textbf{Responsibility:} & \textbf{Collaborators:}\\
        \hline
        Receive input to let SE take exercise & General (Room) Controller\\
        ~ & ~\\
        Receive input to let SE finish exercise & General (Room) Controller\\
        ~ & ~\\
        Show that SE is exercising & ~\\
        \hline
    \end{tabular}
\end{table}

\begin{table}[H]
    \centering
    \begin{tabular}{|p{5cm}|p{5cm}|}
        \hline
        \multicolumn{2}{|l|}{\textbf{Class Name:} Gaming}\\
        \hline
        \textbf{Responsibility:} & \textbf{Collaborators:}\\
        \hline
        Receive input to let SE play games & SE Attributes Controller\\
        ~ & ~\\
        Receive input to let SE stop playing games & SE Attributes Controller\\
        ~ & ~\\
        Show that SE is gaming & ~\\
        \hline
    \end{tabular}
\end{table}

\begin{table}[H]
    \centering
    \begin{tabular}{|p{5cm}|p{5cm}|}
        \hline
        \multicolumn{2}{|l|}{\textbf{Class Name:} Sleeping}\\
        \hline
        \textbf{Responsibility:} & \textbf{Collaborators:}\\
        \hline
        Receive input to let SE play sleep & SE Attributes Controller\\
        ~ & ~\\
        Show that SE is sleeping & ~\\
        \hline
    \end{tabular}
\end{table}

\begin{table}[H]
    \centering
    \begin{tabular}{|p{5cm}|p{5cm}|}
        \hline
        \multicolumn{2}{|l|}{\textbf{Class Name:} Shop UI}\\
        \hline
        \textbf{Responsibility:} & \textbf{Collaborators:}\\
        \hline
        Present item list with prices & Shop Controller\\
        ~ & ~\\
        Receive input to purchase items/skills & Shop Controller\\
        ~ & ~\\
        Present programming skills unowned by SE & Shop Controller\\
        ~ & ~\\
        Present skills information & Shop Controller\\
        ~ & ~\\
        Filter item list & Shop Controller\\
        \hline
    \end{tabular}
\end{table}

\begin{table}[H]
    \centering
    \begin{tabular}{|p{5cm}|p{5cm}|}
        \hline
        \multicolumn{2}{|l|}{\textbf{Class Name:} Project UI}\\
        \hline
        \textbf{Responsibility:} & \textbf{Collaborators:}\\
        \hline
        Present software projects for SE & Project Controller\\
        ~ & ~\\
        Present projects information (profit, skills required, deadline, completion) & Project Controller\\
        ~ & ~\\
        Receive input to let SE start projects & Project Controller\\
        ~ & ~\\
        Receive input to let SE complete projects & Project Controller\\
        \hline
    \end{tabular}
\end{table}

\begin{table}[H]
    \centering
    \begin{tabular}{|p{5cm}|p{5cm}|}
        \hline
        \multicolumn{2}{|l|}{\textbf{Class Name:} Friends Message UI}\\
        \hline
        \textbf{Responsibility:} & \textbf{Collaborators:}\\
        \hline
        Present user’s friends list & Friend and Chat Controller\\
        ~ & ~\\
        Present user current chat & Friend and Chat Controller\\
        ~ & ~\\
        Receive input to send message/chat & Friend and Chat Controller\\
        \hline
    \end{tabular}
\end{table}

\newpage
\appendix
\section{Division of Labour}
\label{sec:division_of_labour}
% Begin Section
Include a Division of Labour sheet which indicates the contributions of each team member. This sheet must be signed by all team members.
% End Section

\newpage
\section*{IMPORTANT NOTES}
\begin{itemize}
%	\item You do \underline{NOT} need to provide a text explanation of each diagram; the diagram should speak for itself
	\item Please document any non-standard notations that you may have used
	\begin{itemize}
		\item \emph{Rule of Thumb}: if you feel there is any doubt surrounding the meaning of your notations, document them
	\end{itemize}
	\item Some diagrams may be difficult to fit into one page
	\begin{itemize}
		\item It is OK if the text is small but please ensure that it is readable when printed
		\item If you need to break a diagram onto multiple pages, please adopt a system of doing so and thoroughly explain how it can be reconnected from one page to the next; if you are unsure about this, please ask about it
	\end{itemize}
	\item Please submit the latest version of Deliverable 1 with Deliverable 2
	\begin{itemize}
		\item It does not have to be a freshly printed version; the latest marked version is OK
	\end{itemize}
	\item If you do \underline{NOT} have a Division of Labour sheet, your deliverable will \underline{NOT} be marked
\end{itemize}


\end{document}
%------------------------------------------------------------------------------