% !TeX root = ./D1.tex
\documentclass[]{article}

% Imported Packages
%------------------------------------------------------------------------------
\usepackage{amssymb}
\usepackage{amstext}
\usepackage{amsthm}
\usepackage{amsmath}
\usepackage{enumerate}
\usepackage{fancyhdr}
\usepackage[margin=1in]{geometry}
\usepackage{graphicx}
\usepackage{extarrows}
\usepackage{setspace}
\usepackage{hyperref}
\hypersetup{
    colorlinks=true,
    linkcolor=blue,
    urlcolor=cyan,
}
\usepackage{float}
\usepackage[shortlabels]{enumitem}
\usepackage{lipsum}
%------------------------------------------------------------------------------

% Header and Footer
%------------------------------------------------------------------------------
\pagestyle{plain}
\renewcommand\headrulewidth{0.4pt}
\renewcommand\footrulewidth{0.4pt}                        
%------------------------------------------------------------------------------

% Title Details
%------------------------------------------------------------------------------
\title{
    \textbf{Group 10 - Deliverable \#1}\\
    \large SFWRENG 3A04: Software Design III - Large System Design
}
\author{
    Andrew Hum 400138826\\
    Arkin Modi 400142497\\
    Hongzhao Tan 400136957\\
    Christopher Vishnu 400129743\\
    Shengchen Zhou 400050783\\
}
\date{Feburary 7, 2020}
%------------------------------------------------------------------------------

% Document
%------------------------------------------------------------------------------
\begin{document}

\pagenumbering{gobble}
\maketitle	
\newpage

\pagenumbering{arabic}
\section{Introduction}
\label{sec:introduction}
This is the Software Requirement Specification for Hacker Sim.

\subsection{Purpose}
\label{sub:purpose}
The purpose of the document is to lay out a description and intention of the 
software that is being developed for both the client and the developers. It 
will convey the software and hardware requirements of the game, as well as the 
design constraints of the software. Furthermore, this document is used to define 
our product, its functionality and performance which will be used to verify the 
product in the testing process. 

\subsection{Scope}
\label{sub:scope}
The purpose of the proposed software, Hacker Sim, will be an enjoyable, 
interactive game that will allow the user to grow their specimen and beat their 
highscores. The focus of the application is to grow a Software Engineer in their 
room. Both the Software Engineer and their room can be upgraded using the 
in-game currency. The in-game currency will be acquired through the completion 
of tasks. Online capabilities will give users the ability to share their rooms 
and chat as well as give gifts to other users to help complete tasks. 


\subsection{Definitions, Acronyms, and Abbreviations}
\label{sub:definitions_acronyms_and_abbreviations}
SE - Software Engineer

\subsection{References}
\label{sub:references}
Dormehl, L. (2019, May 29). The Tamagotchi Effect: How Digital Pets Shaped The 
Way We Use Technology. Retrieved from 
\url{https://www.digitaltrends.com/cool-tech/how-tamagotchi-shaped-tech/}
\\\\
Technologies, U. (n.d.). Unity. Retrieved from \url{https://unity.com/}

\subsection{Overview}
\label{sub:overview}
The document is organized in the following manner: Overall Description, Use 
Case Diagram, Functional Requirements, Non-Functional Requirements. Overall 
Description focuses on providing details about how the ideas link together to 
enable specific functionality and showing functionality through comparison as 
well as outline who the users are intended to be and the constraints. Use Case 
Diagrams demonstrate how the stakeholder carries out business events. Functional 
Requirements explains the functionality of the system. Finally, the 
Non-Functional Requirements focuses on the required qualities of the software.


\section{Overall Description}
\label{sec:overall_description}

\subsection{Product Perspective}
\label{sub:product_perspective}
The product is a system modeling application for desktop computers based on the 
Unity platform using C\#. The product design itself is not a continuation of any 
predecessor products, but in the past there were many implementations with a 
similar concept; raising a digital being. The trend of virtual pet games started 
from mid 1990s to early 2000s, popularized by the Tamagotchis. Our product will 
differentiate itself by giving users an experience of raising a software 
engineer from the very beginning. Our product is a self-contained system that 
will not be interacting with other systems.
\\\\
The system’s main features:
\begin{enumerate}[1)]
    \item A server with functionalities:
    \begin{itemize}
        \item To provide users with online access via login/logout
        \item To connect a database with the implementation of a game
        \item To allow different users to share and interact with each other
        \item To allow users to give feedback and report bugs
    \end{itemize}
    \item A database or a file management system on the user’s computer for 
    storing:
    \begin{itemize}
        \item User account information set
        \item User game progress
        \item Product FAQs for convenient look-up
    \end{itemize}
\end{enumerate}

\subsection{Product Functions}
\label{sub:product_functions}
\subsubsection{Function Summary}
The system’s functionality is centralized around the user raising their digital 
specimen as a Software Engineer (referred to as SE below). The user will be 
involved with the lifecycle of specimen and attempt to promote it with a set 
of means and reach end-game goal. A currency system is designed for the in-game 
shop and the user can earn currency from initiating interactions/events with 
their SE. For example, working and completing coding projects. The in-game 
shop contains three main categories of items: interactable items (exercise 
equipment), furniture and room upgrades and food. The specimen has a list of 
metrics, i.e. tiredness, happiness, age, health, etc. The user must follow 
certain rules to keep metrics within a rational range, otherwise it will 
hinder specimen growth or lead to game failure. The user will be able to view 
their specimen’s information to check their progress and current health. 

\subsubsection{List of Basic Functions}
\begin{enumerate}
    \item User creates a SE.
    \item Users initiate events and interact with their SE.
    \item Users purchase items from the in-game shop.
    \item Users utilize items.
    \item Users customize specimen’s room.
    \item Users view specimen’s information.
    \item Users reset their game stage and start a new one.
\end{enumerate}

\subsubsection{Function Domain}
\begin{figure}[H]
    \centering
    \includegraphics[scale=0.5]{"Function Domain Figure".png}
    \caption{Function Domain}
\end{figure}

\subsection{User Characteristics}
\label{sub:user_characteristics}
Intended users do not have to meet a strict set of requirements to use our 
product. Users can be a PhD student or no education. They are not required to 
have past experience of similar types of games either. In general, it is 
considered that the user should:

\begin{itemize}
    \item Have a hardware to run our product on, i.e. desktop, laptop, etc.
    \item Have basic knowledge of operations for computer with access to internet
    \item Be able to read FAQs and understand how to use the product
\end{itemize}

\subsection{Constraints}
\label{sub:constraints}
Items that might limit the developer's options:

\begin{itemize}
    \item Regulatory policies: Unity has term of service that developers need 
    to follow
    \item Basic requirements expectation set by the project outline: Our 
    product must satisfy these functionalities. Features that have a conflict 
    with those won’t be our options.
    \item Project deliverable deadlines and schedule: In the given limiting 
    time, we may not be able to develop various complex functions that are not 
    included in project outlines with complex implementations.
    \item Developer team’s experience: Some people are using Unity platform for 
    the first time and have finite experience with creating and designing games.
\end{itemize}

\subsection{Assumptions and Dependencies}
\label{sub:assumptions_and_dependencies}
\begin{itemize}
    \item As mentioned above in section 2.3, all requirements are based on the 
    criteria outlined for proper use of the product. If definitions change 
    during development, requirements will change accordingly.
    \item All requirements are written under the assumption that our product is 
    able to run on users’ hardware - the user’s hardware supports the technology 
    of our product.
    \item It is assumed that the product is running in an environment with 
    internet connection.
\end{itemize}

\subsection{Apportioning of Requirements}
\label{sub:apportioning_of_requirements}
Requirements that may be delayed until future versions of the system:

\begin{itemize}
    \item Customization of SE’s room feature might be delayed as it is not a 
    core feature to support main functions of system and it is not a requirement 
    set by project outline
    \item Interaction between SEs of different users feature might be delayed 
    if the team cannot find a proper way to achieve this function in the given 
    time
\end{itemize}

\section{Use Case Diagram}
\label{sec:use_case_diagram}
In this section, we have three Use Case diagrams to illustrate the required 
business events and their interactions with each other and the user.

\subsubsection*{Use Case Diagram 1:}
\begin{figure}[H]
    \centering
    \includegraphics[scale=0.5]{"Use Case Diagram 1".png}
\end{figure}

\begin{enumerate}
    \item Create SE: The user creates his SE, by customizing the basic 
    properties of the SE.
    \item Purchase Items: User purchases in-game items for his SE with in-game 
    currency.
    \item Let the SE Play Games: The user can let his SE play games to gain 
    happiness.
    \item Increase Currency: The user increases his SE’s in-game currency by 
    letting SE work on software projects which would also increase the SE’s 
    level of tiredness.
    \item Accessing Inventory and Using Items: The user accesses the SE’s 
    inventory, adds items into inventory and uses the items inside.
\end{enumerate}

\subsubsection*{Use Case Diagram 2:}
\begin{figure}[H]
    \centering
    \includegraphics[scale=0.5]{"Use Case Diagram 2".png}
\end{figure}

\begin{enumerate}
    \item Using Exercise Equipment: The user lets the SE exercise with 
    its exercise equipment. Letting the SE exercise would keep the SE fit, make 
    the SE tired, and also prolong the SE’s lifespan if the exercise equipment 
    have been used frequently.
    \item Feed The SE: The user shall be able to feed the SE by ordering food 
    in-game. Feeding the SE would decrease the SE’s hunger. 
    \item Overfeed The SE: The user overfeeds the SE by ordering food in-game 
    when the SE is not hungry. Overfeeding the SE could make the SE’s unhealthy.
    \item Improve SE’s skill: The user upgrades the SE’s programming skills, 
    by selecting the skill which the user wants to upgrade, confirming his 
    selection and spending certain amount of SE’s in-game programming experience 
    to complete upgrading. Upgrading programming skills could increase SE’s income.
    \item Communicate With Other Users: The user shall be able to communicate 
    through the system with other users. By communicating with other users, the 
    user can send/receive messages to/from other users, share pictures of the 
    SE’s room and the SE’s current programming skills to other users.
\end{enumerate}

\subsubsection*{Use Case Diagram 3:}
\begin{figure}[H]
    \centering
    \includegraphics[scale=0.5]{"Use Case Diagram 3".png}
\end{figure}

Use-case diagram 3 (references Business Event 5) outlines the interactions 
between the user and the systems reaction to a discrete amount of time passed. 
When a discrete amount of time has passed the system will notify the user of an 
event that may have occurred. These events include:

\begin{itemize}
    \item The SE’s current state (tired, hungry, etc.)
    \item Relevant metrics for the SE
    \item The completion of a current project, and the awarded currency
\end{itemize}

\section{Functional Requirements}
\label{sec:functional_requirements}
The following Business Events are all in the User Viewpoint.


\begin{enumerate}[start=1, label={\textbf{BE\arabic*.}}]
	\item The user wants to create his SE
    \begin{enumerate}[1.]
        \item The system shall allow the user to choose the name of their SE
        \item The system shall allow the user to choose the gender of their SE
        \item The system shall provide the choice to confirm selection
        \item The system shall provide the choice to restart creation
    \end{enumerate}
    
	\item The user wants to purchase items for the SE
	\begin{enumerate}[1.]
		\item The system shall allow the user to browse the in-game store/shop
        \item The user shall be able to purchase furniture, exercise equipment 
        and computer upgrades from the shop
        \item The system shall provide a choice the user to put/remove items 
        into/from his in-game shopping cart and the total price of the items 
        inside the cart changes correspondingly
        \item The system shall allow the user to checkout to purchase all of 
        the items in his in-game shopping cart
        \item The system shall check the user’s current currency
    \end{enumerate}
    
    \item The user wants to increase SE’s in-game currency
	\begin{enumerate}[1.]
		\item The system shall allow the user to open SE’s workspace
        \item The system shall provide a choice for the user to select the 
        project he wants his SE to work on.
        \item The system shall provide the choice to let SE start working
        \item The system shall provide the choice to let SE stop working
        \item Working shall increase the SE’s tiredness level.
    \end{enumerate}
    
    \item Discrete-time step pass in game
	\begin{enumerate}[1.]
        \item The system shall award the user in-game currency and programming 
        experience as the SE completes projects
        \item If the project the SE is currently working on is completed, the 
        system shall let SE stop working and notify the user
        \item The system shall notify the user once any project SE has processed 
        on passed its due date
        \item The system shall show updated aggregation of SE’s important 
        metrics (health, happiness, hunger, age, exercise, tiredness)
        \item The system shall increase the SE’s hunger level overtime
        \item The system shall decrease the SE’s exercise level overtime
        \item The system shall increase the SE’s age overtime
        \item The system shall notify the user once the SE hungry
        \item The system shall notify the user once the SE needs to exercise
        \item The system shall notify the user once the SE needs to rest (SE 
        reaching its maximum tiredness level)
        \item The system shall notify the user once the SE is too unhealthy 
        (close to reaching its end of lifeline)
        \item The system shall notify the user once the SE reaches its end of 
        lifeline
        \item The system shall track the user’s lifespan and record a score
    \end{enumerate}
    
    \item The user wants to access their inventory and use items
	\begin{enumerate}[1.]
        \item The system shall allow the user to open the SE’s inventory display 
        its items
        \item The system shall allow the user to use/consume the items in SE’s 
        inventory
        \item The system shall provide an option to add items to the SE’s 
        inventory        
    \end{enumerate}

    \item The user want to interact with objects in the room
	\begin{enumerate}[1.]
        \item The system shall allow interaction between the SE and its environment
        \item Feeding the pet shall decrease the pet’s hunger level
        \item The system shall allow the SE to use its exercise equipment
        \item Using exercise equipment shall increase SE’s exercise level
        \item Using exercise equipment shall increase SE’s tiredness level
        \item Using exercise equipment frequently shall prolong the SE’s lifespan
        \item The system shall allow the SE to play games on it’s computer
        \item Playing game shall increase the SE’s happiness index
    \end{enumerate}

    \item The user wants to feed the SE
	\begin{enumerate}[1.]
        \item The system shall provide an option to order food
        \item Feeding the SE will reduce its hunger level
        \item Feeding the SE when it is not hungry shall cause it to be overfed
        \item Overfeeding the SE will decrease its health level.        
    \end{enumerate}

    \item The user wants to improve the SE’s skills
	\begin{enumerate}[1.]
        \item The system shall allow the user to choose a programming skill of 
        the SE to upgrade
        \item Upgrading a certain skill shall cost a certain amount of the SE’s 
        programming experience.
        \item Upgrading certain skills shall increase the SE’s income        
    \end{enumerate}

    \item The user wants to communicate with friends
	\begin{enumerate}[1.]
        \item The system shall allow the user to send messages to other users
        \item The system shall allow the user to receive messages from other users
        \item The system shall allow the user to share pictures of their room 
        with other users
        \item The system shall allow the user to share their current programming 
        skills with other users               
    \end{enumerate}
\end{enumerate}

\section{Non-Functional Requirements}
\label{sec:non-functional_requirements}

\subsection{Look and Feel Requirements}
\label{sub:look_and_feel_requirements}

\subsubsection{Appearance Requirements}
\label{ssub:appearance_requirements}
\begin{enumerate}[start=1, label={LF\arabic*.}]
	\item The system shall have a minimalistic design.
\end{enumerate}

\subsubsection{Style Requirements}
\label{ssub:style_requirements}
N/A

\subsection{Usability and Humanity Requirements}
\label{sub:usability_and_humanity_requirements}

\subsubsection{Ease of Use Requirements}
\label{ssub:ease_of_use_requirements}
\begin{enumerate}[start=1, label={UH\arabic*.}]
	\item The user shall face a minimal number of errors. 
	\item The errors shall have a descriptive message.
\end{enumerate}

\subsubsection{Personalization and Internationalization Requirements}
\label{ssub:personalization_and_internationalization_requirements}
\begin{enumerate}[start=3, label={UH\arabic*.}]
	\item The user shall be able to personalize the SE’s name and gender.
\end{enumerate}

\subsubsection{Learning Requirements}
\label{ssub:learning_requirements}
\begin{enumerate}[start=4, label={UH\arabic*.}]
    \item The user shall be able to learn the game mechanics through a quick 
    tutorial.
\end{enumerate}

\subsubsection{Understandability and Politeness Requirements}
\label{ssub:understandability_and_politeness_requirements}
\begin{enumerate}[start=5, label={UH\arabic*.}]
    \item The system icons shall be taken from their common usage icons where 
    applicable. 
    \item The user shall only be promoted with a confirmation box when necessary.
\end{enumerate}

\subsubsection{Accessibility Requirements}
\label{ssub:accessibility_requirements}
\begin{enumerate}[start=7, label={UH\arabic*.}]
	\item The system shall have subtitles under each icon.
\end{enumerate}

\subsection{Performance Requirements}
\label{sub:performance_requirements}

\subsubsection{Speed and Latency Requirements}
\label{ssub:speed_and_latency_requirements}
\begin{enumerate}[start=1, label={PR\arabic*.}]
	\item The system shall respond to a user’s input within 5ms.
\end{enumerate}

\subsubsection{Safety-Critical Requirements}
\label{ssub:safety_critical_requirements}
N/A

\subsubsection{Precision or Accuracy Requirements}
\label{ssub:precision_or_accuracy_requirements}
\begin{enumerate}[start=2, label={PR\arabic*.}]
	\item The system shall record all numerical data upto 8 decimal places.
\end{enumerate}

\subsubsection{Reliability and Availability Requirements}
\label{ssub:reliability_and_availability_requirements}
\begin{enumerate}[start=3, label={PR\arabic*.}]
    \item The system shall be available at all times except for 2 hours a week 
    for maintenance.
\end{enumerate}

\subsubsection{Robustness or Fault-Tolerance Requirements}
\label{ssub:robustness_or_fault_tolerance_requirements}
\begin{enumerate}[start=4, label={PR\arabic*.}]
    \item When network connection during multiplayer is lost, the system will 
    switch to single-player.
\end{enumerate}

\subsubsection{Capacity Requirements}
\label{ssub:capacity_requirements}
\begin{enumerate}[start=5, label={PR\arabic*.}]
	\item The system will allow two users to interact via an online session.
	\item The system will store the player data.
\end{enumerate}

\subsubsection{Scalability or Extensibility Requirements}
\label{ssub:scalability_or_extensibility_requirements}
N/A

\subsubsection{Longevity Requirements}
\label{ssub:longevity_requirements}
\begin{enumerate}[start=7, label={PR\arabic*.}]
    \item The system shall be built and maintained for the duration of the 
    project ending in April 2020.
\end{enumerate}

\subsection{Operational and Environmental Requirements}
\label{sub:operational_and_environmental_requirements}

\subsubsection{Expected Physical Environment}
\label{ssub:expected_physical_environment}
N/A

\subsubsection{Requirements for Interfacing with Adjacent Systems}
\label{ssub:requirements_for_interfacing_with_adjacent_systems}
N/A

\subsubsection{Productization Requirements}
\label{ssub:productization_requirements}
\begin{enumerate}[start=1, label={OE\arabic*.}]
    \item The system shall be distributed via an online repository. 
    \item The system shall be packaged in an executable file (.exe and .app).
\end{enumerate}

\subsubsection{Release Requirements}
\label{ssub:release_requirements}
\begin{enumerate}[start=3, label={OE\arabic*.}]
    \item The system shall have one release at the beginning of April 2020.
\end{enumerate}

\subsection{Maintainability and Support Requirements}
\label{sub:maintainability_and_support_requirements}

\subsubsection{Maintenance Requirements}
\label{ssub:maintenance_requirements}
N/A

\subsubsection{Supportability Requirements}
\label{ssub:supportability_requirements}
\begin{enumerate}[start=1, label={MS\arabic*.}]
    \item The system shall have a tutorial for explanation of the game.
\end{enumerate}

\subsubsection{Adaptability Requirements}
\label{ssub:adaptability_requirements}
N/A

\subsection{Security Requirements}
\label{sub:security_requirements}

\subsubsection{Access Requirements}
\label{ssub:access_requirements}
\begin{enumerate}[start=1, label={SR\arabic*.}]
    \item All project members shall have authorized access to all parts of the 
    system at all times.
\end{enumerate}

\subsubsection{Integrity Requirements}
\label{ssub:integrity_requirements}
N/A

\subsubsection{Privacy Requirements}
\label{ssub:privacy_requirements}
N/A

\subsubsection{Audit Requirements}
\label{ssub:audit_requirements}
N/A

\subsubsection{Immunity Requirements}
\label{ssub:immunity_requirements}
N/A

\subsection{Cultural and Political Requirements}
\label{sub:cultural_and_political_requirements}

\subsubsection{Cultural Requirements}
\label{ssub:cultural_requirements}
N/A

\subsubsection{Political Requirements}
\label{ssub:political_requirements}
N/A

\subsection{Legal Requirements}
\label{sub:legal_requirements}

\subsubsection{Compliance Requirements}
\label{ssub:compliance_requirements}
N/A

\subsubsection{Standards Requirements}
\label{ssub:standards_requirements}
N/A

\newpage
\appendix
\section{Division of Labour}
\label{sec:division_of_labour}
\begin{table}[H]
    \centering
    \caption{Division of Labour}
    \begin{tabular}{|p{2.5cm}|p{2.5cm}|p{10.5cm}|}
        \hline
        \textbf{Section} & \textbf{Contributor(s)} & \textbf{Description}\\
        \hline
        \textbf{Brainstorming} & All Members & Developed an idea for the project 
        and answered the minimum requirement questions\\
        \hline
        \textbf{Introduction} & Christopher Vishnu & Completed and revised 
        the introduction\\
        \hline
        \textbf{Overall Description} & Shengchen Zhou, Andrew Hum & Shengchen 
        completed all the sections in the Overall Description. Andrew revised 
        the sections and checked for grammar and consistency.\\
        \hline
        \textbf{Functional Requirements} & Andrew Hum, Hongzhao Tan & Both 
        contributors developed and revised all the Business Events and 
        Functional Requirements\\
        \hline
        \textbf{Non-Functional Requirements} & Arkin Modi & Completed and 
        revised all the non-functional requirements\\
        \hline
    \end{tabular}
\end{table}

\vspace{2cm}

\begin{table}[H]
    \begin{tabular}{p{5cm}}
    \\
    \hline
    Andrew Hum
    \\\\\\\\
    \hline
    Arkin Modi
    \\\\\\\\
    \hline
    Hongzhao Tan
    \\\\\\\\
    \hline
    Christopher Vishnu
    \\\\\\\\
    \hline
    Shengchen Zhou
    \end{tabular}
\end{table}

\end{document}
%------------------------------------------------------------------------------