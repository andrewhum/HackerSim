% !TeX root = ./D1.tex
\documentclass[]{article}

% Imported Packages
%------------------------------------------------------------------------------
\usepackage{amssymb}
\usepackage{amstext}
\usepackage{amsthm}
\usepackage{amsmath}
\usepackage{enumerate}
\usepackage{fancyhdr}
\usepackage[margin=1in]{geometry}
\usepackage{graphicx}
\usepackage{extarrows}
\usepackage{setspace}
\usepackage{hyperref}
\hypersetup{
    colorlinks=true,
    linkcolor=blue,
    urlcolor=cyan,
}
\usepackage{float}
%------------------------------------------------------------------------------

% Header and Footer
%------------------------------------------------------------------------------
\pagestyle{plain}  
\renewcommand\headrulewidth{0.4pt}                                      
\renewcommand\footrulewidth{0.4pt}                                    
%------------------------------------------------------------------------------

% Title Details
%------------------------------------------------------------------------------
\title{Deliverable \#1 Template}
\author{SE 3A04: Software Design II -- Large System Design}
\date{}                               
%------------------------------------------------------------------------------

% Document
%------------------------------------------------------------------------------
\begin{document}

\maketitle	

\section{Introduction}
\label{sec:introduction}
This is the Software Requirement Specification for Hacker Sim.

\subsection{Purpose}
\label{sub:purpose}
The purpose of the document is to lay out a description and intention of the 
software that is being developed for both the client and the developers. It 
will convey the software and hardware requirements of the game, as well as the 
design constraints of the software. Furthermore, this document is used to define 
our product, its functionality and performance which will be used to verify the 
product in the testing process. 

\subsection{Scope}
\label{sub:scope}
The purpose of the proposed software, Hacker Sim, will be an enjoyable, 
interactive game that will allow the user to grow their specimen and beat their 
highscores. The focus of the application is to grow a Software Engineer in their 
room. Both the Software Engineer and their room can be upgraded using the 
in-game currency. The in-game currency will be acquired through the completion 
of tasks. Online capabilities will give users the ability to share their rooms 
and chat as well as give gifts to other users to help complete tasks. 


\subsection{Definitions, Acronyms, and Abbreviations}
\label{sub:definitions_acronyms_and_abbreviations}
SE - Software Engineer

\subsection{References}
\label{sub:references}
Dormehl, L. (2019, May 29). The Tamagotchi Effect: How Digital Pets Shaped The 
Way We Use Technology. Retrieved from 
\url{https://www.digitaltrends.com/cool-tech/how-tamagotchi-shaped-tech/}
\\\\
Technologies, U. (n.d.). Unity. Retrieved from \url{https://unity.com/}

\subsection{Overview}
\label{sub:overview}
The document is organized in the following manner: Overall Description, Use 
Case Diagram, Functional Requirements, Non-Functional Requirements. Overall 
Description focuses on providing details about how the ideas link together to 
enable specific functionality and showing functionality through comparison as 
well as outline who the users are intended to be and the constraints. Use Case 
Diagrams demonstrate how the stakeholder carries out business events. Functional 
Requirements explains the functionality of the system. Finally, the 
Non-Functional Requirements focuses on the required qualities of the software.


\section{Overall Description}
\label{sec:overall_description}

\subsection{Product Perspective}
\label{sub:product_perspective}
The product is a system modeling application for desktop computers based on the 
Unity platform using C\#. The product design itself is not a continuation of any 
predecessor products, but in the past there were many implementations with a 
similar concept; raising a digital being. The trend of virtual pet games started 
from mid 1990s to early 2000s, popularized by the Tamagotchis. Our product will 
differentiate itself by giving users an experience of raising a software engineer 
from the very beginning.

The system’s main features:
\begin{enumerate}[1)]
    \item A server with functionalities:
    \begin{itemize}
        \item To provide users with online access via login/logout
        \item To connect a database with the implementation of a game
        \item To allow different users to share and interact with each other
        \item To allow users to give feedback and report bugs
    \end{itemize}
    \item A database or a file management system on the user’s computer for 
    storing:
    \begin{itemize}
        \item User account information set
        \item User game progress
        \item Product FAQs for convenient look-up
    \end{itemize}
\end{enumerate}

\subsection{Product Functions}
\label{sub:product_functions}
\subsubsection{Function Summary}
The system’s functionality is centralized around the user raising their digital 
specimen as a Software Engineer (referred to as SE below). The user will be 
involved with the lifecycle of specimen and attempt to promote it with a set 
of means and reach end-game goal. A currency system is designed for the in-game 
shop and the user can earn currency from initiating interactions/events with 
their SE. For example, working and completing coding projects. The in-game 
shop contains three main categories of items: interactable items (pets and 
exercise equipment), furniture and room upgrades and food. The specimen has a 
list of metrics, i.e. tiredness, happiness, age, health, etc. The user must 
follow certain rules to keep metrics within a rational range, otherwise it will 
hinder specimen growth or lead to game failure. The user will be able to view 
their specimen’s information to check their progress and current health. 

\subsubsection{List of Basic Functions}
\begin{enumerate}
    \item User creates a SE.
    \item Users initiate events and interact with their SE.
    \item Users purchase items from the in-game shop.
    \item Users utilize items.
    \item Users customize specimen’s room.
    \item Users view specimen’s information.
    \item Users reset their game stage and start a new one.
\end{enumerate}

\subsubsection{Function Domain}
\begin{figure}[H]
    \centering
    \includegraphics[scale=0.5]{"Function Domain Figure".png}
    \caption{Function Domain}
\end{figure}

\subsection{User Characteristics}
\label{sub:user_characteristics}
Intended users do not have to meet a strict set of requirements to use our 
product. Users can be a PhD student or no education. They are not required to 
have past experience of similar types of games either. In general, it is 
considered that the user should:

\begin{itemize}
    \item Have a hardware to run our product on, i.e. desktop, laptop, etc.
    \item Have basic knowledge of operations for computer with access to internet
    \item Be able to read FAQs and understand how to use the product
\end{itemize}

\subsection{Constraints}
\label{sub:constraints}
Items that might limit the developer's options:

\begin{itemize}
    \item Regulatory policies: Unity has term of service that developers need 
    to follow
    \item Basic requirements expectation set by the project outline: Our 
    product must satisfy these functionalities. Features that have a conflict 
    with those won’t be our options.
    \item Project deliverable deadlines and schedule: In the given limiting 
    time, we may not be able to develop various complex functions that are not 
    included in project outlines with complex implementations.
    \item Developer team’s experience: Some people are using Unity platform for 
    the first time and have finite experience with creating and designing games.
\end{itemize}

\subsection{Assumptions and Dependencies}
\label{sub:assumptions_and_dependencies}
\begin{itemize}
    \item As mentioned above in section 2.3, all requirements are based on the 
    criteria outlined for proper use of the product. If definitions change 
    during development, requirements will change accordingly.
    \item All requirements are written under the assumption that our product is 
    able to run on users’ hardware - the user’s hardware supports the technology 
    of our product.
    \item It is assumed that the product is running in an environment with 
    internet connection.
\end{itemize}

\subsection{Apportioning of Requirements}
\label{sub:apportioning_of_requirements}
Requirements that may be delayed until future versions of the system:

\begin{itemize}
    \item Customization of SE’s room feature might be delayed as it is not a 
    core feature to support main functions of system and it is not a requirement 
    set by project outline
    \item Interaction between SEs of different users feature might be delayed 
    if the team cannot find a proper way to achieve this function in the given 
    time
\end{itemize}

\section{Use Case Diagram}
\label{sec:use_case_diagram}
% Begin Section
This section should provide a use case diagram for your application. 
\begin{enumerate}[a)]
	\item Each use case appearing in the diagram should be accompanied by a text description. 
\end{enumerate}
% End Section

\section{Functional Requirements}
\label{sec:functional_requirements}
% Begin Section
This section of the SRS should contain all of the software requirements to a level of detail sufficient to enable designers to design a system to satisfy those requirements, and testers to test that the system satisfies those requirements. Throughout this section, every stated requirement should be externally perceivable by users, operators, or other external systems. These requirements should include at a minimum a description of every input (stimulus) into the system, every output (response) from the system, and all functions performed by the system in response to an input or in support of an output.

You normally have two options for organizing your functional requirements:
\begin{enumerate}
	\item Organize first by \emph{business events}, then by \emph{viewpoints}
	\item Organize first by \emph{viewpoints}, then by \emph{business events}
\end{enumerate}
Choose the one which makes the most sense.

For example, if you wish to organization by business events:
\begin{enumerate}[{BE}1.]
	\item Business Event
	\begin{enumerate}[{VP1}.1]
		\item Viewpoint
			\begin{enumerate}
				\item Requirement
				\item Requirement
				\item \dots
			\end{enumerate}
		\item Viewpoint
			\begin{enumerate}
				\item Requirement
				\item Requirement
				\item \dots
			\end{enumerate}
		\item \dots
	\end{enumerate}
	\item Business Event
	\begin{enumerate}[{VP2}.1]
		\item Viewpoint
			\begin{enumerate}
				\item Requirement
				\item Requirement
				\item \dots
			\end{enumerate}
		\item Viewpoint
			\begin{enumerate}
				\item Requirement
				\item Requirement
				\item \dots
			\end{enumerate}
		\item \dots
	\end{enumerate}
\end{enumerate}

\underline{OR}, if you wish to organization by viewpoints:
\begin{enumerate}[{VP}1.]
	\item Viewpoint 
	\begin{enumerate}[{BE1}.1]
		\item Business Event
		\begin{enumerate}
			\item Requirement
			\item Requirement
			\item \dots
		\end{enumerate}
		\item Business Event
		\begin{enumerate}
			\item Requirement
			\item Requirement
			\item \dots
		\end{enumerate}
		\item \dots
	\end{enumerate}
	\item Viewpoint
	\begin{enumerate}[{BE2}.1]
		\item Business Event
		\begin{enumerate}
			\item Requirement
			\item Requirement
			\item \dots
		\end{enumerate}
		\item Business Event
		\begin{enumerate}
			\item Requirement
			\item Requirement
			\item \dots
		\end{enumerate}
		\item \dots
	\end{enumerate}
\end{enumerate}

% End Section

\section{Non-Functional Requirements}
\label{sec:non-functional_requirements}
% Begin Section
\subsection{Look and Feel Requirements}
\label{sub:look_and_feel_requirements}
% Begin SubSection

\subsubsection{Appearance Requirements}
\label{ssub:appearance_requirements}
% Begin SubSubSection
\begin{enumerate}[{LF}1. ]
	\item 
\end{enumerate}
% End SubSubSection

\subsubsection{Style Requirements}
\label{ssub:style_requirements}
% Begin SubSubSection
\begin{enumerate}[{LF}1. ]
	\item 
\end{enumerate}
% End SubSubSection

% End SubSection

\subsection{Usability and Humanity Requirements}
\label{sub:usability_and_humanity_requirements}
% Begin SubSection

\subsubsection{Ease of Use Requirements}
\label{ssub:ease_of_use_requirements}
% Begin SubSubSection
\begin{enumerate}[{UH}1. ]
	\item 
\end{enumerate}
% End SubSubSection

\subsubsection{Personalization and Internationalization Requirements}
\label{ssub:personalization_and_internationalization_requirements}
% Begin SubSubSection
\begin{enumerate}[{UH}1. ]
	\item 
\end{enumerate}
% End SubSubSection

\subsubsection{Learning Requirements}
\label{ssub:learning_requirements}
% Begin SubSubSection
\begin{enumerate}[{UH}1. ]
	\item 
\end{enumerate}
% End SubSubSection

\subsubsection{Understandability and Politeness Requirements}
\label{ssub:understandability_and_politeness_requirements}
% Begin SubSubSection
\begin{enumerate}[{UH}1. ]
	\item 
\end{enumerate}
% End SubSubSection

\subsubsection{Accessibility Requirements}
\label{ssub:accessibility_requirements}
% Begin SubSubSection
\begin{enumerate}[{UH}1. ]
	\item 
\end{enumerate}
% End SubSubSection

% End SubSection

\subsection{Performance Requirements}
\label{sub:performance_requirements}
% Begin SubSection

\subsubsection{Speed and Latency Requirements}
\label{ssub:speed_and_latency_requirements}
% Begin SubSubSection
\begin{enumerate}[{PR}1. ]
	\item 
\end{enumerate}
% End SubSubSection

\subsubsection{Safety-Critical Requirements}
\label{ssub:safety_critical_requirements}
% Begin SubSubSection
\begin{enumerate}[{PR}1. ]
	\item 
\end{enumerate}
% End SubSubSection

\subsubsection{Precision or Accuracy Requirements}
\label{ssub:precision_or_accuracy_requirements}
% Begin SubSubSection
\begin{enumerate}[{PR}1. ]
	\item 
\end{enumerate}
% End SubSubSection

\subsubsection{Reliability and Availability Requirements}
\label{ssub:reliability_and_availability_requirements}
% Begin SubSubSection
\begin{enumerate}[{PR}1. ]
	\item 
\end{enumerate}
% End SubSubSection

\subsubsection{Robustness or Fault-Tolerance Requirements}
\label{ssub:robustness_or_fault_tolerance_requirements}
% Begin SubSubSection
\begin{enumerate}[{PR}1. ]
	\item 
\end{enumerate}
% End SubSubSection

\subsubsection{Capacity Requirements}
\label{ssub:capacity_requirements}
% Begin SubSubSection
\begin{enumerate}[{PR}1. ]
	\item 
\end{enumerate}
% End SubSubSection

\subsubsection{Scalability or Extensibility Requirements}
\label{ssub:scalability_or_extensibility_requirements}
% Begin SubSubSection
\begin{enumerate}[{PR}1. ]
	\item 
\end{enumerate}
% End SubSubSection

\subsubsection{Longevity Requirements}
\label{ssub:longevity_requirements}
% Begin SubSubSection
\begin{enumerate}[{PR}1. ]
	\item 
\end{enumerate}
% End SubSubSection

% End SubSection

\subsection{Operational and Environmental Requirements}
\label{sub:operational_and_environmental_requirements}
% Begin SubSection

\subsubsection{Expected Physical Environment}
\label{ssub:expected_physical_environment}
% Begin SubSubSection
\begin{enumerate}[{OE}1. ]
	\item 
\end{enumerate}
% End SubSubSection

\subsubsection{Requirements for Interfacing with Adjacent Systems}
\label{ssub:requirements_for_interfacing_with_adjacent_systems}
% Begin SubSubSection
\begin{enumerate}[{OE}1. ]
	\item 
\end{enumerate}
% End SubSubSection

\subsubsection{Productization Requirements}
\label{ssub:productization_requirements}
% Begin SubSubSection
\begin{enumerate}[{OE}1. ]
	\item 
\end{enumerate}
% End SubSubSection

\subsubsection{Release Requirements}
\label{ssub:release_requirements}
% Begin SubSubSection
\begin{enumerate}[{OE}1. ]
	\item 
\end{enumerate}
% End SubSubSection

% End SubSection

\subsection{Maintainability and Support Requirements}
\label{sub:maintainability_and_support_requirements}
% Begin SubSection

\subsubsection{Maintenance Requirements}
\label{ssub:maintenance_requirements}
% Begin SubSubSection
\begin{enumerate}[{MS}1. ]
	\item 
\end{enumerate}
% End SubSubSection

\subsubsection{Supportability Requirements}
\label{ssub:supportability_requirements}
% Begin SubSubSection
\begin{enumerate}[{MS}1. ]
	\item 
\end{enumerate}
% End SubSubSection

\subsubsection{Adaptability Requirements}
\label{ssub:adaptability_requirements}
% Begin SubSubSection
\begin{enumerate}[{MS}1. ]
	\item 
\end{enumerate}
% End SubSubSection

% End SubSection

\subsection{Security Requirements}
\label{sub:security_requirements}
% Begin SubSection

\subsubsection{Access Requirements}
\label{ssub:access_requirements}
% Begin SubSubSection
\begin{enumerate}[{SR}1. ]
	\item 
\end{enumerate}
% End SubSubSection

\subsubsection{Integrity Requirements}
\label{ssub:integrity_requirements}
% Begin SubSubSection
\begin{enumerate}[{SR}1. ]
	\item 
\end{enumerate}
% End SubSubSection

\subsubsection{Privacy Requirements}
\label{ssub:privacy_requirements}
% Begin SubSubSection
\begin{enumerate}[{SR}1. ]
	\item 
\end{enumerate}
% End SubSubSection

\subsubsection{Audit Requirements}
\label{ssub:audit_requirements}
% Begin SubSubSection
\begin{enumerate}[{SR}1. ]
	\item 
\end{enumerate}
% End SubSubSection

\subsubsection{Immunity Requirements}
\label{ssub:immunity_requirements}
% Begin SubSubSection
\begin{enumerate}[{SR}1. ]
	\item 
\end{enumerate}
% End SubSubSection

% End SubSection

\subsection{Cultural and Political Requirements}
\label{sub:cultural_and_political_requirements}
% Begin SubSection

\subsubsection{Cultural Requirements}
\label{ssub:cultural_requirements}
% Begin SubSubSection
\begin{enumerate}[{CP}1. ]
	\item 
\end{enumerate}
% End SubSubSection

\subsubsection{Political Requirements}
\label{ssub:political_requirements}
% Begin SubSubSection
\begin{enumerate}[{CP}1. ]
	\item 
\end{enumerate}
% End SubSubSection

% End SubSection

\subsection{Legal Requirements}
\label{sub:legal_requirements}
% Begin SubSection

\subsubsection{Compliance Requirements}
\label{ssub:compliance_requirements}
% Begin SubSubSection
\begin{enumerate}[{LR}1. ]
	\item 
\end{enumerate}
% End SubSubSection

\subsubsection{Standards Requirements}
\label{ssub:standards_requirements}
% Begin SubSubSection
\begin{enumerate}[{LR}1. ]
	\item 
\end{enumerate}
% End SubSubSection

% End SubSection

% End Section

\appendix
\section{Division of Labour}
\label{sec:division_of_labour}
% Begin Section
Include a Division of Labour sheet which indicates the contributions of each team member. This sheet must be signed by all team members.
% End Section

\newpage
\section*{IMPORTANT NOTES}
\begin{itemize}
	\item Be sure to include all sections of the template in your document regardless whether you have something to write for each or not
	\begin{itemize}
		\item If you do not have anything to write in a section, indicate this by the \emph{N/A}, \emph{void}, \emph{none}, etc.
	\end{itemize}
	\item Uniquely number each of your requirements for easy identification and cross-referencing
	\item Highlight terms that are defined in Section~1.3 (\textbf{Definitions, Acronyms, and Abbreviations}) with \textbf{bold}, \emph{italic} or \underline{underline}
	\item For Deliverable 1, please highlight, in some fashion, all (you may have more than one) creative and innovative features. Your creative and innovative features will generally be described in Section~2.2 (\textbf{Product Functions}), but it will depend on the type of creative or innovative features you are including.
\end{itemize}


\end{document}
%------------------------------------------------------------------------------